\documentclass[portrait, a0b,final]{a0poster}%landscape
\usepackage{epsfig}
%\usepackage{pdf}
\usepackage{graphicx}
\usepackage{multicol}
\usepackage{pstricks,pst-grad}
\usepackage[spanish,activeacute]{babel}
\usepackage[latin1]{inputenc}
\usepackage{amsthm,amssymb,amsmath,amsfonts}
\numberwithin{equation}{section}
\newtheorem{theorem}{Teorema}[section]



%%%%%%%%%%%%%%%%%%%%%%%%%%%%%%%%%%%%%%%%%%%
% Definition of some variables and colors
%\renewcommand{\rho}{\varrho}
%\renewcommand{\phi}{\varphi}
\setlength{\columnsep}{3cm} \setlength{\columnseprule}{2mm}
\setlength{\parindent}{0.0cm}

%%%%%%%%%%%%%%%%%%%%%%%%%%%%%%%%%%%%%%%%%%%%%%%%%%%%
%%%              UAM-Logo en Latex               %%%
%%%         Por Ismael Vel�zquez Ram�rez         %%%
%%%%%%%%%%%%%%%%%%%%%%%%%%%%%%%%%%%%%%%%%%%%%%%%%%%%

\newcommand{\uamlogo}[3][2pt]{
    \psset{unit=#2,linewidth=#1 }
    \psline*[linearc=.25,linecolor=#3](2.8,2)(2,2)(1.8,0)(2.8,2)(3.8,0)(3.6,2)(2,2)(1.8,0)
    \psline*[linecolor=#3](0,0)(.8,0)(1.8,2)(1,2)(0,0)
    \psline*[linecolor=#3](4.8,0)(3.8,2)(4.6,2)(5.6,0)(4.8,0)
    \psline*[linearc=.25,linecolor=#3](3.8,0)(2.8,2)(3.6,2)(4.6,0)(3.8,0)
    \psline*[linearc=.25,linecolor=#3](4.6,0)(3.8,0)(2.8,2)(3.6,2)(4.6,0)
    \rput{180}(5.6,2){%
    \psline*[linearc=.25,linecolor=white](2.8,2)(2,2)(1.8,0)(2.8,2)(3.8,0)(3.6,2)(2,2)(1.8,0)
    \psline*[linearc=.25,linecolor=white](1,0)(1.8,0)(2.8,2)(2,2)(1,0)
    \psline*[linearc=.25,linecolor=white](1.8,0)(1,0)(2,2)(2.8,2)(1.8,0)
    \psline*[linearc=.25,linecolor=white](3.8,0)(2.8,2)(3.6,2)(4.6,0)(3.8,0)
    \psline*[linearc=.25,linecolor=white](4.6,0)(3.8,0)(2.8,2)(3.6,2)(4.6,0)
    \psline[linearc=.25,linecolor=#3](1,0)(2,2)(3.6,2)(4.6,0)
    \psline[linecolor=#3](1,0)(1.8,0)
    \psline[linearc=.25,linecolor=#3](4.6,0)(3.8,0)
    \psline[linearc=.25,linecolor=#3](1.8,0)(2.8,2)(3.8,0)}
    \psline*[linearc=.25,linecolor=#3](1,0)(1.8,0)(2.8,2)(2,2)(1,0)
    \psline*[linearc=.25,linecolor=#3](1.8,0)(1,0)(2,2)(2.8,2)(1.8,0)}

%%%%%%%%%%%%%%%%%%%%%%%%%%%%%%%%%%%%%%%%%%%%%%%%%%%%
%%%               Background                     %%%
%%%%%%%%%%%%%%%%%%%%%%%%%%%%%%%%%%%%%%%%%%%%%%%%%%%%

\newcommand{\background}[3]{
  \newrgbcolor{cgradbegin}{#1}
  \newrgbcolor{cgradend}{#2}
  \psframe[fillstyle=gradient,gradend=cgradend,
  gradbegin=cgradbegin,gradmidpoint=#3](0.,0.)(1.\textwidth,-1.\textheight)
}



%%%%%%%%%%%%%%%%%%%%%%%%%%%%%%%%%%%%%%%%%%%%%%%%%%%%
%%%                Poster                        %%%
%%%%%%%%%%%%%%%%%%%%%%%%%%%%%%%%%%%%%%%%%%%%%%%%%%%%

\newenvironment{poster}{
  \begin{center}
  \begin{minipage}[c]{0.98\textwidth}
}{
  \end{minipage}
  \end{center}
}



%%%%%%%%%%%%%%%%%%%%%%%%%%%%%%%%%%%%%%%%%%%%%%%%%%%%
%%%                pcolumn                       %%%
%%%%%%%%%%%%%%%%%%%%%%%%%%%%%%%%%%%%%%%%%%%%%%%%%%%%

\newenvironment{pcolumn}[1]{
  \begin{minipage}{#1\textwidth}
  \begin{center}
}{
  \end{center}
  \end{minipage}
}



%%%%%%%%%%%%%%%%%%%%%%%%%%%%%%%%%%%%%%%%%%%%%%%%%%%%
%%%                pbox                          %%%
%%%%%%%%%%%%%%%%%%%%%%%%%%%%%%%%%%%%%%%%%%%%%%%%%%%%

\newrgbcolor{lcolor}{0.04 0.58 0.29}
\newrgbcolor{gcolor1}{1. 1. 1.}
\newrgbcolor{gcolor2}{.80 .80 1.}

\newcommand{\pbox}[4]{
\psshadowbox[#3]{
\begin{minipage}[t][#2][t]{#1}
#4
\end{minipage}
}}



%%%%%%%%%%%%%%%%%%%%%%%%%%%%%%%%%%%%%%%%%%%%%%%%%%%%
%%%                myfig                         %%%
%%%%%%%%%%%%%%%%%%%%%%%%%%%%%%%%%%%%%%%%%%%%%%%%%%%%
% \myfig - replacement for \figure
% necessary, since in multicol-environment
% \figure won't work

\newcommand{\myfig}[3][0]{
\begin{center}
  \vspace{1.5cm}
  \includegraphics[width=#3\hsize,angle=#1]{#2}
  \nobreak\medskip
\end{center}}



%%%%%%%%%%%%%%%%%%%%%%%%%%%%%%%%%%%%%%%%%%%%%%%%%%%%
%%%                mycaption                     %%%
%%%%%%%%%%%%%%%%%%%%%%%%%%%%%%%%%%%%%%%%%%%%%%%%%%%%
% \mycaption - replacement for \caption
% necessary, since in multicol-environment \figure and
% therefore \caption won't work

%\newcounter{figure}
\setcounter{figure}{1}
\newcommand{\mycaption}[1]{
  \vspace{0.5cm}
  \begin{quote}
    {{\sc Figura} \arabic{figure}: #1}
  \end{quote}
  \vspace{1cm}
  \stepcounter{figure}
}

%%%%%%%%%%%%%%%%%%%%%%%%%%%%%%%%%%%%%%%%%%%%%%%%%%%%%%%%%%%%%%%%%%%%%%
%%%  INICIO DEL DOCUMENTO
%%%%%%%%%%%%%%%%%%%%%%%%%%%%%%%%%%%%%%%%%%%%%%%%%%%%%%%%%%%%%%%%%%%%%%

\begin{document}

\background{1.00 0.92 0.59}{1. 1. 1.}{0.1}

\vspace*{2cm}

\definecolor{rojo}{rgb}{0.96,0.45,0.18}
\newrgbcolor{lightblue}{1.00 0.50 0.00}
\newrgbcolor{white}{1. 1. 1.}
%\newrgbcolor{whiteblue}{1.00 0.63 0.26}
\newrgbcolor{whiteblue}{1 1 1}

\begin{poster}
\language1
%%%------------------------------------------------------------------------------------
%%%                                        INICIA ENCABEZADO
%%%------------------------------------------------------------------------------------
\begin{center}
\begin{pcolumn}{0.98}

\pbox{0.95\textwidth}{}{linewidth=2mm,framearc=0.3,linecolor=lightblue,fillstyle=gradient,gradangle=0,gradbegin=white,gradend=whiteblue,gradmidpoint=1.0,framesep=1em}{

%%%----------------------------------------------------------------------------------------------------------
%%%                                    LOGO 10� ANIVERSARIO UAM
%%%----------------------------------------------------------------------------------------------------------
\begin{minipage}[c][9.5cm][c]{0.15\textwidth}                                % 9.5cm 0.15\tex...
  \begin{center}
        \begin{center}
           \myfig{logogrande.ps}{1.0}    
%           \mycaption{ LOGO UAM 10� ANIVERSARIO}
        \end{center}
  \end{center}
\end{minipage}
%%%---------------------------------------------------------------------------------------------------------
%%%                                    T�TULO DEL TRABAJO
%%%----------------------------------------------------------------------------------------------------------
\begin{minipage}[c][9.5cm][c]{0.80\textwidth}                                 % 0.65
  \begin{center}
    {\sc {\textcolor{rojo}{\Huge Congreso internacional de}}}\\[4mm]
    {\sc {\textcolor{rojo}{\Huge Matem\'aticas y sus aplicaciones.}} }\\[8mm]
    {\sc \Huge El modelo de Ising aplicado a la ciencia pol\'itica}\\[4mm]
    {\Large Juan Manuel Romero Sanpedro* y \'Angel C\'aceres Licona**} \\
    {\large *Departamento de Matem\'aticas Aplicadas, veri@correo.cua.uam.mx , **Alumno de Matem\'aticas Aplicadas, angelcaceres@outlook.com} \\%[2mm]
  \end{center}
\end{minipage}
%-----------------------------------------------------------------------------------------------
%\begin{minipage}[c][9.5cm][c]{0.15\textwidth}                                % 9.5cm 0.15\tex...
%  \begin{center}
%    \rput(-4,-1){\uamlogo{2}{rojo}                                          % (-5,-2)
%    \rput[tl](-.2,-.3){\textcolor{rojo}{\Large Casa abierta al \ tiempo}}   % (-.2,-.3)
%    \rput[tl](3.6,-.8){Cuajimalpa}}                                         % (3.6,-.8)
%  \end{center}
%\end{minipage}
%%%%%%--------------------------------------------------------------------------------------------------------
%%                                      TERMINA ENCABEZADO
%%%%%%--------------------------------------------------------------------------------------------------------


}
\end{pcolumn}
\end{center}


\vspace*{1cm}



%%%%%%%%%%%%%%%%%%%%%
%%% Content
%%%%%%%%%%%%%%%%%%%%%
\begin{center}
%%%------------------------------------------------------------------------------------
%%%%                                     INICIA COLUMNA 1
%%%------------------------------------------------------------------------------------
\begin{pcolumn}{0.32}
\pbox{0.9\textwidth}{95cm}{linewidth=2mm,framearc=0.1,linecolor=lightblue,fillstyle=gradient,gradangle=0,gradbegin=white,gradend=white,gradmidpoint=1.0,framesep=1em}{

%%%%%%%%--------------------------- SECCI�N 1 -----------------------------------------------
    \begin{center}
        \pbox{0.8\textwidth}{}%%
        {linewidth=2mm,framearc=0.1,linecolor=lightblue,fillstyle=gradient,gradangle=0,%%
        gradbegin=white,gradend=whiteblue,gradmidpoint=1.0,framesep=1em}{
        \begin{center}
            Abstract
        \end{center}}
    \end{center}
    \vspace{1.25cm}
%%%%%%%%--------------------------- SECCI�N 1 -----------------------------------------------

El modelo de Ising es un sistema de la F\'isica Estad\'istica el cual se ha usado  para entender diversos fen\'omenos  sociales, como conflictos b\'elicos y comerciales. En en este trabajo se muestra una analog\'ia entre el modelo de Ising y la teor\'ia de juegos. Adem\'as este modelo se usa para estudiar un tratado de libre comercio entre dos pa\'ises. En este ejemplo, los pa\'ises pueden escoger entre seguir pol\'iticas proteccionistas o de libre mercado.


%%%%%%%%--------------------------- SECCI�N 2 -----------------------------------------------
    \vspace{2cm}
    \begin{center}
        \pbox{0.8\textwidth}{}%%
        {linewidth=2mm,framearc=0.1,linecolor=lightblue,fillstyle=gradient,gradangle=0,%%
        gradbegin=white,gradend=whiteblue,gradmidpoint=1.0,framesep=1em}{%%
        \begin{center}
            Introducci\'{o}n
        \end{center}}
    \end{center}
    \vspace{1.25cm}
%%%%%%%%--------------------------- SECCI�N 2 -----------------------------------------------

    El modelo de Ising es un modelo f\'isico originalmente propuesto para estudiar el comportamiento de materiales ferromagn\'eticos.
    El modelo de Ising fue propuesto por Wilhelm Lenz que lo concibi\'o como un problema para su alumno Ernst Ising. Ising logr\'o resolver el modelo unidimensional para su tesis de 1924 y el modelo bidimensional fue resuelto por Lars Onsager hasta 1944. En el modelo de Ising tenemos $N$ part\'iculas en una matriz cuadrada. Cada part\'icula puede esyat apuntando hacia arriba o hacia abajo y a cada una de esas orientaciones se le llama \textit{esp\'in de la part\'icula}. El sentido de este esp\'in es determinado por la interacc\'on de la part\'icula con sus vecinas. \\

    El modelo de Ising es uno de los pocos modelos de part\'iculas interactuantes para el cual se conoce una soluci\'on exacta. Es de gran utilidad ya que, aunque originalmente fue formulado para resolver problemas f\'isicos (ferromagnetismo) tiene much\'isima aplicaciones en el modelado de problemas de otras \'areas como la biolog\'ia, finanzas, etc como se mostrar\'a.\\

  

%%%%%%%%--------------------------- SECCI�N 3 -----------------------------------------------
    \vspace{2cm}
    \begin{center}
        \pbox{0.8\textwidth}{}%%
        {linewidth=2mm,framearc=0.1,linecolor=lightblue,fillstyle=gradient,gradangle=0,%%
        gradbegin=white,gradend=whiteblue,gradmidpoint=1.0,framesep=1em}{%%
        \begin{center}
            Modelo de Ising
        \end{center}}
    \end{center}
    \vspace{1.25cm}
    \setcounter{section}{3}%
\setcounter{equation}{0}%
%%%%%%%%--------------------------- SECCI�N 3 -----------------------------------------------

En una dimensi\'on, el Hamiltoniano del modelo de Ising puede ser escrito como \\
%
\begin{eqnarray}
  \mathbb{H}=-\epsilon\sum_{i=1}^{N}\sigma_i\sigma_{i+1}-\mu B\sum_{i=1}^{N}\sigma_i \label{hamilIsing}
\end{eqnarray}
%
donde $\sigma=\pm1$ y estos valores indican cada uno de los estados posibles: Si la part\'icula apunta hacia arriba o hacia abajo. Se usa tambi\'en la siguiente representaci\'on matricial:
%
\begin{eqnarray}
|\uparrow\rangle &=&\begin{bmatrix}1\\0\end{bmatrix}\label{spinup},\\
|\downarrow\rangle&=&\begin{bmatrix}0\\1\end{bmatrix}\label{spindown},
\end{eqnarray}
%
y se considera que la red es c\'iclica, es decir:
%
$$
\sigma_{N}=\sigma_{N+1},
$$
%
lo cual equivale a resolver el problema en un anillo.\\
Una funci\'on muy \'util para describir el sistema, y que a partir de ella se pueden calcular propiedades del sistema como la energ\'ia libre, temperatura del sistema, etc. En el caso del modelo de Ising la funci\'on de partici\'on est\'a dada por: 
%
\begin{eqnarray}
  Z_{N}(T,B)=\sum_{\sigma_{1}=\pm 1}\cdots\sum_{\sigma_{N=\pm 1}}e^{\beta\sum{i=1^N\left[\epsilon\sigma_{i}\sigma_{i+1}+\frac{1}{2}\mu \beta (\sigma_{i}\sigma{i+1}) \right]}} .\label{partising}
\end{eqnarray}
%
La funci\'on de partici\'on puede ser escrita en t\'erminos de la siguiente matriz, que es conocida como matriz de transferencia:
%
\begin{eqnarray}
\bar P= \begin{pmatrix}
e^{\beta(\epsilon + \mu \beta )} && e^{-\beta \epsilon} \\
e^{-\beta \epsilon} && e^{\beta(\epsilon - \mu \beta )}
\end{pmatrix},\label{matrizP}
\end{eqnarray}
%
Se puede mostrar, de manera an\'alitica que
%
\begin{eqnarray}
Z_N(T,B)= Tr(\bar P^N).
\end{eqnarray}
%
Un resultado muy \'util para calcular propiedades del sistema. Por ejemplo, si calculamos los valores propios:
%
\begin{eqnarray}
\lambda_{\pm}=e^{\beta \epsilon}\left[\cosh(\beta\mu B)\pm\sqrt{\cosh^2(\beta\mu B)-2e{-2\beta\epsilon\sinh(2\beta\epsilon)}}\right],
\end{eqnarray}
%
de ah\'i podemos obtener la energ\'ia libre:
%
\begin{eqnarray}
g(T,B)=-\frac{1}{\beta}\log{e^{\beta \epsilon} \cosh(\beta B) + \left[e^{2\beta \epsilon }\cosh^2 (\beta B) - 2\sinh (2\beta B)\right]^{\frac{1}{2}}}.
\end{eqnarray}
%


   }
\end{pcolumn}
%%%--------------------------------------------------------------------------------
%%%              TERMINA COLUMNA 1 E INICIA COLUMNA 2
%%%--------------------------------------------------------------------------------
\begin{pcolumn}{0.32}
\pbox{0.9\textwidth}{95cm}{linewidth=2mm,framearc=0.1,linecolor=lightblue,fillstyle=gradient,gradangle=0,gradbegin=white,gradend=white,gradmidpoint=1.0,framesep=1em}{

  La magnetizaci\'on por spin est\'a dada por 
  %
  \begin{eqnarray}
  m(T,B)=-\frac{\partial g}{\partial B} = \frac{\sinh(\beta B)}{[\sin^2(\beta B)+ e^{-4 B\epsilon}]\frac{1}{2}}.
  \end{eqnarray}
  %

  \begin{center}    
    \myfig{elementosDeDominio.ps}{.75}
    \mycaption{Elementos de dominio en modelo de ising bidimensional.}
  \end{center}


%%%%%%%%--------------------------- SECCI�N 4 -----------------------------------------------
\vspace{2cm}
    \begin{center}
        \pbox{0.8\textwidth}{}%%
        {linewidth=2mm,framearc=0.1,linecolor=lightblue,fillstyle=gradient,gradangle=0,%%
        gradbegin=white,gradend=whiteblue,gradmidpoint=1.0,framesep=1em}{%%
        \begin{center}
           Teor\'ia de Juegos 
        \end{center}}
    \end{center}
    \vspace{1.25cm}
\setcounter{section}{4}%
\setcounter{equation}{0}%
%%%%%%%%--------------------------- SECCI�N 4 -----------------------------------------------
La teor\'ia de juegos estudia modelos matem\'aticos de conflicto y cooperaci\'on entre tomadores de decisiones racionales. Los problemas en teor\'ia de juegos son descritos generalmente con $N$ jugadores que tienen un conjunto $s_x = {1,2,4,...,N}$ estrategias disponibles. Cada jugador adoptar\'a una estrategia que maximizar\'a su ganancia $u_x$ en el siguiente paso. En casos especiales existe un estado estacionario en el que a ning\'un jugador le favorece cambiar de estrategia. Matem\'aticamente este estado satisface la siguiente condici\'on.
%
\begin{eqnarray}
  u_x \{s_1^*, s_2^*, ..., s_N^*\} \leq u_x\{s_1^*, s_2^*, ...s_x',..., s_N^*\} \qquad \forall x, \forall s_x' \neq s_x*.
\end{eqnarray}%
Esto se conoce como el equilibrio de Nash y cuando la desigualdad es estricta se le llama equilibrio puro de Nash.\\

Un problema b\'asico en teor\'ia de juegos es aquel en el que se tienen dos jugadores y dos estrategias: Cooperaci\'on (C) y deserci\'on (D). La ganancia en esta situaci\'on puede ser representada usando esta matriz:
%
\begin{eqnarray}
\begin{bmatrix}
  C & D
\end{bmatrix} \\ \nonumber
\begin{bmatrix}
  C\\
  D
\end{bmatrix}\nonumber
\begin{bmatrix}
  R & S\\
  T & P
\end{bmatrix}\label{matrizNash}
\end{eqnarray}
%
donde la fila $(C,D)$ demnota las opciones de estrategia del jugador que estamos intentando determinar y la columna $(C, D)$ las de el otro jugador. $R$ es la recompensa obtenida cuando ambos cooperan, $S$ es el costo que se paga por un jugador cuando \'este coopera y el otro no, $T$ es la ganancia por desertar cuando el otro jugador si coopera y $P$ es el castigo que se paga cuando ambos jugadores desertan.
En un sistema probabil\'istico con $N$ participantes el inter\'es principal es saber cu\'antos participantes desertan comparado con el n\'umero de participantes. Esto est\'a dado por: 
%
\begin{eqnarray}
m=\frac{P_C - P_D}{N}, \label{mJuegos}
\end{eqnarray}
%
En este sistema, al convertir a os agentes en part\'iculas, cada uno de las estrategias se convierte en un estado del esp\'in: $\sigma = +1$ para la cooperaci\'on y $\sigma = -1$ para la deserci\'on.\\
La ecuaci\'on (\ref{mJuegos}) se convierte en la ecuaci\'on de la magnetizaci\'on promedio del sistema. \\
Para \'esto, Sarkar y Benjamin \cite{benjamin}  desarrollaron el siguiente me\'etodo.\\
La premisa es que el equilibrio de Nash se mantiene intacto si transformamos la matriz (\ref{matrizNash}) de esta forma:
%
\begin{eqnarray}
  U =
  \begin{bmatrix}
    R & S\\
    T & P
  \end{bmatrix} \to 
  \begin{bmatrix}
    R - \lambda & S - \mu \\
    T - \lambda & P - \mu
  \end{bmatrix} = U'
\end{eqnarray}
%
EL Hamiltoniano para este sistema est\'a dado por: 
%
\begin{eqnarray}
  \mathbb{H} = -\sum_{i=1}^{N} J\sigma_{i} \sigma_{i+1} - \sum_{i=1}^{n}\frac{h}{2} (\sigma_{i} \sigma_{i+1}).
\end{eqnarray}
%
La funci\'on de partici\'on est\'a dada por 
%
\begin{eqnarray}
  Z= e^{N \beta J}(\cosh(\beta h) + \sqrt{\sinh^2(\beta h) + e^{-4 \beta J}}).
\end{eqnarray}
%
}
\end{pcolumn}
%%%----------------------------------------------------------------------------------
%%%              TERMINA COLUMNA 2 E INICIA COLUMNA 3
%%%----------------------------------------------------------------------------------
\begin{pcolumn}{0.32}
\pbox{0.9\textwidth}{95cm}{linewidth=2mm,framearc=0.1,linecolor=lightblue,fillstyle=gradient,gradangle=0,gradbegin=white,gradend=white,gradmidpoint=1.0,framesep=1em}{

%%%---------------------------- SECCI�N 5 ---------------------------------------------
\begin{center}
        \pbox{0.8\textwidth}{}%%
        {linewidth=2mm,framearc=0.1,linecolor=lightblue,fillstyle=gradient,gradangle=0,%%
        gradbegin=white,gradend=whiteblue,gradmidpoint=1.0,framesep=1em}{
        \begin{center}
Aplicaciones:      un tratado bilateral de libre comercio
        \end{center}}
\end{center}    
    \vspace{1.25cm}
\setcounter{section}{5}%
\setcounter{equation}{0}%
%%%---------------------------- SECCI�N 6 ---------------------------------------------

Muchas situaciones en la ciancia pol\'itica pueden ser vistas como agentes jugando el mismo juego una y otra vez. Es de inter\'es estudiar como ciertas pr\'acticas, convenciones y cooperaci\'on se sostienen cuando los involucrados pueden tener alg\'un inventivo en el corto plazo por desviarse de el comportamiento esperado. Por ejemplo: Los tratados de libre comercio. Muchas veces se cree que la econom\'ia global mejorar\'ia si todos los pa\'ises accedieran a el libre comercio, pero que individulmeante les ir\'ia mejor si adoptan medidas proteccionistas. Por ejemplo, se considera la siguiente representaci\'on de las pol\'iticas de comercio entre M\'exico y Estados Unidos:
%
\begin{eqnarray}
  \begin{tabular}{ |c|c|c| } 
   \hline
   MX/EU & Libre Mercado & Proteccionismo \\ 
   \hline
   Libre Mercado & 10,10 & 1,12 \\ 
   \hline
   Proteccionismo & 12,1 & 4,4 \\ 
   \hline
  \end{tabular} 
\end{eqnarray}
%
En este caso, si el juego es jugado s\'olo una vez, el equilibrio de Nash se da cuando los dos pa\'ises eligen las pol\'iticas proteccionistas. Al generalizar para un n\'umero infinito de juegos podemos ver que se repite el patr\'on en el que cada pa\'is elige las pol\'iticas proteccionistas.


%%%---------------------------- SECCI�N 6 ---------------------------------------------
\vspace{2cm}
    \begin{center}
        \pbox{0.8\textwidth}{}%%
        {linewidth=2mm,framearc=0.1,linecolor=lightblue,fillstyle=gradient,gradangle=0,%%
        gradbegin=white,gradend=whiteblue,gradmidpoint=1.0,framesep=1em}{%%
        \begin{center}
            Conclusiones
        \end{center}}
    \end{center}
%%%%------------------------------------------------------------------------------------------    
Vemos que el modelo de Ising es muy \'util para resolver problemas incluso en las ciencias sociales.


%%%-----------------------------------------------------------------------------------
%%%                          Bibliograf�a
%%%------------------------------------------------------------------------------------
\bibliographystyle{alpha}
\bibliography{add}

AAA
}
\end{pcolumn}
\end{center}




\end{poster}

\end{document}
