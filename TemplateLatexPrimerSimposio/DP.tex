\documentclass[portrait, a0b,final]{a0poster}%landscape
\usepackage{epsfig}
%\usepackage{pdf}
\usepackage{graphicx}
\usepackage{multicol}
\usepackage{pstricks,pst-grad}
\usepackage[spanish,activeacute]{babel}
\usepackage[latin1]{inputenc}
\usepackage{amsthm,amssymb,amsmath,amsfonts}
\numberwithin{equation}{section}
\newtheorem{theorem}{Teorema}[section]



%%%%%%%%%%%%%%%%%%%%%%%%%%%%%%%%%%%%%%%%%%%
% Definition of some variables and colors
%\renewcommand{\rho}{\varrho}
%\renewcommand{\phi}{\varphi}
\setlength{\columnsep}{3cm} \setlength{\columnseprule}{2mm}
\setlength{\parindent}{0.0cm}

%%%%%%%%%%%%%%%%%%%%%%%%%%%%%%%%%%%%%%%%%%%%%%%%%%%%
%%%              UAM-Logo en Latex               %%%
%%%         Por Ismael Vel�zquez Ram�rez         %%%
%%%%%%%%%%%%%%%%%%%%%%%%%%%%%%%%%%%%%%%%%%%%%%%%%%%%

\newcommand{\uamlogo}[3][2pt]{
    \psset{unit=#2,linewidth=#1 }
    \psline*[linearc=.25,linecolor=#3](2.8,2)(2,2)(1.8,0)(2.8,2)(3.8,0)(3.6,2)(2,2)(1.8,0)
    \psline*[linecolor=#3](0,0)(.8,0)(1.8,2)(1,2)(0,0)
    \psline*[linecolor=#3](4.8,0)(3.8,2)(4.6,2)(5.6,0)(4.8,0)
    \psline*[linearc=.25,linecolor=#3](3.8,0)(2.8,2)(3.6,2)(4.6,0)(3.8,0)
    \psline*[linearc=.25,linecolor=#3](4.6,0)(3.8,0)(2.8,2)(3.6,2)(4.6,0)
    \rput{180}(5.6,2){%
    \psline*[linearc=.25,linecolor=white](2.8,2)(2,2)(1.8,0)(2.8,2)(3.8,0)(3.6,2)(2,2)(1.8,0)
    \psline*[linearc=.25,linecolor=white](1,0)(1.8,0)(2.8,2)(2,2)(1,0)
    \psline*[linearc=.25,linecolor=white](1.8,0)(1,0)(2,2)(2.8,2)(1.8,0)
    \psline*[linearc=.25,linecolor=white](3.8,0)(2.8,2)(3.6,2)(4.6,0)(3.8,0)
    \psline*[linearc=.25,linecolor=white](4.6,0)(3.8,0)(2.8,2)(3.6,2)(4.6,0)
    \psline[linearc=.25,linecolor=#3](1,0)(2,2)(3.6,2)(4.6,0)
    \psline[linecolor=#3](1,0)(1.8,0)
    \psline[linearc=.25,linecolor=#3](4.6,0)(3.8,0)
    \psline[linearc=.25,linecolor=#3](1.8,0)(2.8,2)(3.8,0)}
    \psline*[linearc=.25,linecolor=#3](1,0)(1.8,0)(2.8,2)(2,2)(1,0)
    \psline*[linearc=.25,linecolor=#3](1.8,0)(1,0)(2,2)(2.8,2)(1.8,0)}

%%%%%%%%%%%%%%%%%%%%%%%%%%%%%%%%%%%%%%%%%%%%%%%%%%%%
%%%               Background                     %%%
%%%%%%%%%%%%%%%%%%%%%%%%%%%%%%%%%%%%%%%%%%%%%%%%%%%%

\newcommand{\background}[3]{
  \newrgbcolor{cgradbegin}{#1}
  \newrgbcolor{cgradend}{#2}
  \psframe[fillstyle=gradient,gradend=cgradend,
  gradbegin=cgradbegin,gradmidpoint=#3](0.,0.)(1.\textwidth,-1.\textheight)
}



%%%%%%%%%%%%%%%%%%%%%%%%%%%%%%%%%%%%%%%%%%%%%%%%%%%%
%%%                Poster                        %%%
%%%%%%%%%%%%%%%%%%%%%%%%%%%%%%%%%%%%%%%%%%%%%%%%%%%%

\newenvironment{poster}{
  \begin{center}
  \begin{minipage}[c]{0.98\textwidth}
}{
  \end{minipage}
  \end{center}
}



%%%%%%%%%%%%%%%%%%%%%%%%%%%%%%%%%%%%%%%%%%%%%%%%%%%%
%%%                pcolumn                       %%%
%%%%%%%%%%%%%%%%%%%%%%%%%%%%%%%%%%%%%%%%%%%%%%%%%%%%

\newenvironment{pcolumn}[1]{
  \begin{minipage}{#1\textwidth}
  \begin{center}
}{
  \end{center}
  \end{minipage}
}



%%%%%%%%%%%%%%%%%%%%%%%%%%%%%%%%%%%%%%%%%%%%%%%%%%%%
%%%                pbox                          %%%
%%%%%%%%%%%%%%%%%%%%%%%%%%%%%%%%%%%%%%%%%%%%%%%%%%%%

\newrgbcolor{lcolor}{0.04 0.58 0.29}
\newrgbcolor{gcolor1}{1. 1. 1.}
\newrgbcolor{gcolor2}{.80 .80 1.}

\newcommand{\pbox}[4]{
\psshadowbox[#3]{
\begin{minipage}[t][#2][t]{#1}
#4
\end{minipage}
}}



%%%%%%%%%%%%%%%%%%%%%%%%%%%%%%%%%%%%%%%%%%%%%%%%%%%%
%%%                myfig                         %%%
%%%%%%%%%%%%%%%%%%%%%%%%%%%%%%%%%%%%%%%%%%%%%%%%%%%%
% \myfig - replacement for \figure
% necessary, since in multicol-environment
% \figure won't work

\newcommand{\myfig}[3][0]{
\begin{center}
  \vspace{1.5cm}
  \includegraphics[width=#3\hsize,angle=#1]{#2}
  \nobreak\medskip
\end{center}}



%%%%%%%%%%%%%%%%%%%%%%%%%%%%%%%%%%%%%%%%%%%%%%%%%%%%
%%%                mycaption                     %%%
%%%%%%%%%%%%%%%%%%%%%%%%%%%%%%%%%%%%%%%%%%%%%%%%%%%%
% \mycaption - replacement for \caption
% necessary, since in multicol-environment \figure and
% therefore \caption won't work

%\newcounter{figure}
\setcounter{figure}{1}
\newcommand{\mycaption}[1]{
  \vspace{0.5cm}
  \begin{quote}
    {{\sc Figura} \arabic{figure}: #1}
  \end{quote}
  \vspace{1cm}
  \stepcounter{figure}
}

%%%%%%%%%%%%%%%%%%%%%%%%%%%%%%%%%%%%%%%%%%%%%%%%%%%%%%%%%%%%%%%%%%%%%%
%%%  INICIO DEL DOCUMENTO
%%%%%%%%%%%%%%%%%%%%%%%%%%%%%%%%%%%%%%%%%%%%%%%%%%%%%%%%%%%%%%%%%%%%%%

\begin{document}

\background{1.00 0.92 0.59}{1. 1. 1.}{0.1}

\vspace*{2cm}

\definecolor{rojo}{rgb}{0.96,0.45,0.18}
\newrgbcolor{lightblue}{1.00 0.50 0.00}
\newrgbcolor{white}{1. 1. 1.}
%\newrgbcolor{whiteblue}{1.00 0.63 0.26}
\newrgbcolor{whiteblue}{1 1 1}

\begin{poster}
\language1
%%%------------------------------------------------------------------------------------
%%%                                        INICIA ENCABEZADO
%%%------------------------------------------------------------------------------------
\begin{center}
\begin{pcolumn}{0.98}

\pbox{0.95\textwidth}{}{linewidth=2mm,framearc=0.3,linecolor=lightblue,fillstyle=gradient,gradangle=0,gradbegin=white,gradend=whiteblue,gradmidpoint=1.0,framesep=1em}{

%%%----------------------------------------------------------------------------------------------------------
%%%                                    LOGO 10� ANIVERSARIO UAM
%%%----------------------------------------------------------------------------------------------------------
\begin{minipage}[c][9.5cm][c]{0.15\textwidth}                                % 9.5cm 0.15\tex...
  \begin{center}
        \begin{center}
           \myfig{10cua_variacion1.eps}{1.0}    
%           \mycaption{ LOGO UAM 10� ANIVERSARIO}
        \end{center}
  \end{center}
\end{minipage}
%%%---------------------------------------------------------------------------------------------------------
%%%                                    T�TULO DEL TRABAJO
%%%----------------------------------------------------------------------------------------------------------
\begin{minipage}[c][9.5cm][c]{0.80\textwidth}                                 % 0.65
  \begin{center}
    {\sc {\textcolor{rojo}{\Huge Primer Simposio de las Licenciaturas de la Divisi�n de}}}\\[4mm]
    {\sc {\textcolor{rojo}{\Huge Ciencias Naturales e Ingenier�a}} }\\[8mm]
    {\sc \Huge Aplicaci\'{o}n de la Mec\'{a}nica Cu\'{a}ntica a las Finanzas}\\[4mm]
    {\Large Juan Diego Hern\'andez } \\
    {\large Departamento de Matem�ticas Aplicadas, werlix@outlook.com* } \\%[2mm]
  \end{center}
\end{minipage}
%-----------------------------------------------------------------------------------------------
%\begin{minipage}[c][9.5cm][c]{0.15\textwidth}                                % 9.5cm 0.15\tex...
%  \begin{center}
%    \rput(-4,-1){\uamlogo{2}{rojo}                                          % (-5,-2)
%    \rput[tl](-.2,-.3){\textcolor{rojo}{\Large Casa abierta al \ tiempo}}   % (-.2,-.3)
%    \rput[tl](3.6,-.8){Cuajimalpa}}                                         % (3.6,-.8)
%  \end{center}
%\end{minipage}
%%%%%%--------------------------------------------------------------------------------------------------------
%%                                      TERMINA ENCABEZADO
%%%%%%--------------------------------------------------------------------------------------------------------


}
\end{pcolumn}
\end{center}


\vspace*{1cm}



%%%%%%%%%%%%%%%%%%%%%
%%% Content
%%%%%%%%%%%%%%%%%%%%%
\begin{center}
%%%------------------------------------------------------------------------------------
%%%%                                     INICIA COLUMNA 1
%%%------------------------------------------------------------------------------------
\begin{pcolumn}{0.32}
\pbox{0.9\textwidth}{95cm}{linewidth=2mm,framearc=0.1,linecolor=lightblue,fillstyle=gradient,gradangle=0,gradbegin=white,gradend=white,gradmidpoint=1.0,framesep=1em}{

%%%%%%%%--------------------------- SECCI�N 1 -----------------------------------------------
    \begin{center}
        \pbox{0.8\textwidth}{}%%
        {linewidth=2mm,framearc=0.1,linecolor=lightblue,fillstyle=gradient,gradangle=0,%%
        gradbegin=white,gradend=whiteblue,gradmidpoint=1.0,framesep=1em}{
        \begin{center}
           Abstract
        \end{center}}
    \end{center}
    \vspace{1.25cm}
%%%%%%%%--------------------------- SECCI�N 1 -----------------------------------------------

  
Poner el   abstract enviado  



%%%%%%%%--------------------------- SECCI�N 2 -----------------------------------------------
    \vspace{2cm}
    \begin{center}
        \pbox{0.8\textwidth}{}%%
        {linewidth=2mm,framearc=0.1,linecolor=lightblue,fillstyle=gradient,gradangle=0,%%
        gradbegin=white,gradend=whiteblue,gradmidpoint=1.0,framesep=1em}{%%
        \begin{center}
            Introducci\'{o}n
        \end{center}}
    \end{center}
    \vspace{1.25cm}
%%%%%%%%--------------------------- SECCI�N 2 -----------------------------------------------

    
    
Bachelier,  \\

 Mandelbrot \\
    
      }
\end{pcolumn}
%%%--------------------------------------------------------------------------------
%%%              TERMINA COLUMNA 1 E INICIA COLUMNA 2
%%%--------------------------------------------------------------------------------
\begin{pcolumn}{0.32}
\pbox{0.9\textwidth}{95cm}{linewidth=2mm,framearc=0.1,linecolor=lightblue,fillstyle=gradient,gradangle=0,gradbegin=white,gradend=white,gradmidpoint=1.0,framesep=1em}{


%%%%%%%%--------------------------- SECCI�N 3 -----------------------------------------------
    \vspace{2cm}
    \begin{center}
        \pbox{0.8\textwidth}{}%%
        {linewidth=2mm,framearc=0.1,linecolor=lightblue,fillstyle=gradient,gradangle=0,%%
        gradbegin=white,gradend=whiteblue,gradmidpoint=1.0,framesep=1em}{%%
        \begin{center}
        Ecuaciones de Schr�dinger y de Black -Scholes -Merton
        \end{center}}
    \end{center}
    \vspace{1.25cm}
    \setcounter{section}{3}%
\setcounter{equation}{0}%
%%%%%%%%--------------------------- SECCI�N 3 -----------------------------------------------
Tomamos la ecuaci�n de Schr�dinger:
$$i\hbar\frac{\partial\Psi}{\partial\tilde{t}}=-\frac{\hbar^2}{2m}\frac{\partial^2\Psi}{\partial x^2}$$
Si nombramos $\tilde{t}=it,~\sigma^2=\frac{1}{m}, ~\hbar=1$ nuestra ecuaci�n toma la forma:
%
\begin{eqnarray}
\frac{\partial \Psi}{\partial t}=-\frac{\sigma^2}{2}\frac{\partial^2\Psi}{\partial x^2}.\label{eq:I2}
\end{eqnarray}
%
Aqu\'i, $\sigma$ es la volatilidad del activo suyacente.  Usando esta versi�n de la ecuaci�n de Schr�dinger, veremos que podemos obtener la
ecuaci\'on de Black-Scholes. Primero consideremos las definiciones
%
\begin{eqnarray}
\Psi=e^\vartriangle C(x,t),\quad \vartriangle=-[\frac{1}{\sigma^2}(\frac{\sigma^2}{2}-r)x+\frac{1}{2\sigma^2}(\frac{\sigma^2}{2}+r)^2 t],
\end{eqnarray}
%
donde $r$ es la tasa de inter\'es anualizada y $C$ es el precio de una opci\'on. Usando a $\Psi$ encontramos que 
%
\begin{eqnarray}
\frac{\partial \Psi}{\partial t}&=& e^\vartriangle \left[-\frac{1}{2\sigma^2}(\frac{\sigma^2}{2}+r)^2C(x,t)+\frac{\partial C(x,t)}{\partial t}\right],\qquad \label{eq:I3}\\
\frac{\partial^2 \Psi}{\partial t^2}&=& e^\vartriangle \Bigg[\frac{1}{4\sigma^4} 
\left( \frac{\sigma^2}{2}+r \right)^2
  - \frac{1}{\sigma^2}
  \left(\frac{\sigma^2}{2}+r\right)^2
    \frac{\partial }{\partial t} +\frac{\partial^2 }{\partial t^2}\Bigg]C(x,t),  \qquad \\
     \frac{\partial \Psi}{\partial x}&=&e^\vartriangle 
\left[ -\frac{1}{\sigma^2}(\frac{\sigma^2}{2}-r) C(x,t) + \frac{\partial C(x,t)}{\partial x}
\right],\\
\frac{\partial^2 \Psi}{\partial x^2}&=&e^\vartriangle \Bigg[\frac{1}{\sigma^4}\left(\frac{\sigma^2}{2}-r\right)^2  -\frac{2}{\sigma^2} \left(\frac{\sigma^2}{2}-r\right) \frac{\partial }{\partial x} + \frac{\partial^2 }{\partial x^2}\Bigg]C(x,t).\qquad \label{eq:I1}
\end{eqnarray}
%
Ahora usaremos el cambio de variable 
$$x= \log S, \qquad S= e^x.$$
Usamos  $S$ para representar  el precio del activo subyacente.  Si $f$ es una funci\'on, aplicando la regla de la cadena  obtenemos 
%
\begin{eqnarray}
& &\frac{\partial f}{\partial x}=S\frac{\partial f}{\partial S},\\
& &\frac{\partial^2 f}{\partial x^2}=S\frac{\partial f}{\partial S}+S^2\frac{\partial^2 f}{\partial S^2} .
\end{eqnarray}
%

Usando estos resultados en (\ref{eq:I1}) se llega a 
%
\begin{eqnarray}
\frac{\partial^2 \Psi}{\partial x^2}=e^\vartriangle \left[\frac{1}{\sigma^4}\left(\frac{\sigma^2}{2}-r\right)^2  C +\frac{2r}{\sigma^2}S\frac{\partial C}{\partial S}+ S^2 \frac{\partial^2 C}{\partial S^2}\right]
\end{eqnarray}
%
Considerando esta \'ultima ecuaci\'on y (\ref{eq:I3}) en la ecuaci\'on (\ref{eq:I2}) se encuentra 
%
\begin{eqnarray}
& &e^\vartriangle \left[-\frac{1}{2\sigma^2}(\frac{\sigma^2}{2}+r)^2C+\frac{\partial C}{\partial t}\right]\nonumber\\
& &=-\frac{\sigma^2}{2}e^\vartriangle \left[\frac{1}{\sigma^4}\left(\frac{\sigma^2}{2}-r\right)^2  C +\frac{2r}{\sigma^2}S\frac{\partial C}{\partial S}+ S^2 \frac{\partial^2 C}{\partial S^2}\right].
\end{eqnarray}
%
Esta ecuaci\'on se puede escribir de la forma
%
\begin{eqnarray}
\frac{\partial C(S,t)}{\partial t}=-\frac{\sigma^2}{2}S^2\frac{\partial^2 C(S,t)}{\partial S^2}-rS\frac{\partial C(S,t)}{\partial S}+rC,
\end{eqnarray}
%
la cual es la ecuaci\'on de Black-Scholes. Para opciones Europeas se tiene la soluci\'on 
%
\begin{eqnarray}
C(S,t)=  SN(d_{+})-Ke^{-r(T-t)} N(d_{-}) ,
\end{eqnarray}
%  
que es la f\'ormula de Black-Scholes.
%%%%%%%%--------------------------- SECCI�N 4 -----------------------------------------------
\vspace{1cm}
    \begin{center}
        \pbox{0.8\textwidth}{}%%
        {linewidth=2mm,framearc=0.1,linecolor=lightblue,fillstyle=gradient,gradangle=0,%%
        gradbegin=white,gradend=whiteblue,gradmidpoint=1.0,framesep=1em}{%%
        \begin{center}
            USDMXN
        \end{center}}
    \end{center}
    \vspace{1.25cm}
\setcounter{section}{4}%
\setcounter{equation}{0}%
%%%%%%%%--------------------------- SECCI�N 4 -----------------------------------------------




}
\end{pcolumn}
%%%----------------------------------------------------------------------------------
%%%              TERMINA COLUMNA 2 E INICIA COLUMNA 3
%%%----------------------------------------------------------------------------------
\begin{pcolumn}{0.32}
\pbox{0.9\textwidth}{95cm}{linewidth=2mm,framearc=0.1,linecolor=lightblue,fillstyle=gradient,gradangle=0,gradbegin=white,gradend=white,gradmidpoint=1.0,framesep=1em}{

Ahora, usando el mapeo

   
%%%---------------------------- SECCI�N 5 ---------------------------------------------
\vspace{2cm}
\begin{center}
        \pbox{0.8\textwidth}{}%%
        {linewidth=2mm,framearc=0.1,linecolor=lightblue,fillstyle=gradient,gradangle=0,%%
        gradbegin=white,gradend=whiteblue,gradmidpoint=1.0,framesep=1em}{
        \begin{center}
            Conclusiones
        \end{center}}
\end{center}    
    \vspace{1.25cm}
\setcounter{section}{5}%
\setcounter{equation}{0}%
%%%---------------------------- SECCI�N 6 ---------------------------------------------

RESUMEN 
%%%---------------------------- SECCI�N 6 ---------------------------------------------

%%%---------------------------- SECCI�N 6 ---------------------------------------------
\vspace{2cm}
    \begin{center}
        \pbox{0.8\textwidth}{}%%
        {linewidth=2mm,framearc=0.1,linecolor=lightblue,fillstyle=gradient,gradangle=0,%%
        gradbegin=white,gradend=whiteblue,gradmidpoint=1.0,framesep=1em}{%%
        \begin{center}
          Referencias
        \end{center}}
    \end{center}
%%%%------------------------------------------------------------------------------------------    

%%%-----------------------------------------------------------------------------------
%%%                          Bibliograf�a
%%%------------------------------------------------------------------------------------


\bibliographystyle{alpha}
\bibliography{add}

% \bibitem{black}
[1] F. Black and  M. Scholes, {\it The pricing options and corporate liabilities,} Journal of Political Economy  {\bf 81},  637 (1973).\\
 
[2] R.C. Merton, {\it Theory of Rational Option Pricing,}  Bell J. Econ. and Management Sci.   {\bf 4},  141 (1973).\\

[ 3] B. E. Baaquie,  {\it Quantum Finance,} Cambridge University Press (2004).


}
\end{pcolumn}
\end{center}




\end{poster}

\end{document}
